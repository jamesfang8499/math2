\chapter{实验几何}

几何学是研究“空间”的形体和性质的科学。“空间”
就是我们和万物以至星象天体共存的所在。在日常生活中,
我们举目四望所见到的地方,都是空间的一部分。同学们在
小学数学课中学过的柱体、锥体、球体等等,它们都各自占
有空间的一部分,并且构成不同的形体。各种形体的种种性
质,如各部分的长度、角度、面积,以及体积等等。都是在
我们的生活和生产实践中所不可缺少的知识。
\begin{figure}[htp]
	\centering
\begin{tikzpicture}[scale=.7]
\begin{scope}
	\draw (1.5,0) ellipse [x radius=1.5, y radius=.5];
	\shade (0,0) rectangle (3,4);
	\shade (1.5,4)[draw] ellipse [x radius=1.5, y radius=.5];
	\draw (0,0)--(0,4);
	\draw (3,0)--(3,4);
		\node at (1.5,-1.5){圆柱};

\end{scope}
\begin{scope}[xshift=5.5cm]	
	\shade (1.5,0)[draw] ellipse [x radius=1.5, y radius=.5];
		\fill[gray!50] (0,0)-- (3,0)--(1.5,4)--(0,0);
		\draw (0,0)--(1.5,4)-- (3,0);
	

	\node at (1.5,-1.5){圆锥};
\end{scope}
\begin{scope}[xshift=12cm]
		\shade [draw, ball color =gray!20] (0,2) circle(2);
	\node at (0,-1.5){球};
\end{scope}
\end{tikzpicture}	
	\caption{}
\end{figure}

自古以来,人们经过实践、观察、分析,已总结出一
系列的有关空间方面的知识,例如,从中国、埃及、巴比
伦、玛雅等古文明中,可以看出对空间的知识都已掌握得
相当丰富了。对于空间知识有系统的研究,从西方的古文
明中可知,起始于古埃及和巴比仑,而在古希腊得到蓬勃
的发展,获得较辉煌的成就。大体说来,古希腊在空间知
识方面的成就,由欧几里得集其大成于他所著的《几
何原
本》\footnote{欧几里得(Euclid约公元前300年左右)所著此书原名Elements, 
	我国明代数学家徐光启(公元1562---1633)把书中部分几何内容
	译成中文定名为“几何原本”。“几何学”这个中文的名称即来源于
	此。}这部书中。在这部书里,欧儿里得把当时所知道的几何
知识经过整理,建立起一个初步完整的理论体系,使这部书反
映出几何学是一门偏重于推理、论证的高度理论性的科学。
但是,和任何其它科学一样,几何学的理论基础也是建
立在实验所得的一些基本事实之上的。在这一章里,我们就
通过实验、观察、归纳来研究所得到的知识,为以后进一
步学习论证几何作准备。

\section{点、直线和平面}
点、直线和平面是空间最简单的,也是最基本的图形。
同学们在日常生活中,对它们早已有直观的认识了。在这一
节里,我们再对它们的本质和相互关系作进一步的分析,确
立点、直线和平面这三个基本的几何概念,并总结点、直线
和平面之间相互关系方面的一些基本性质。

\subsection{点和直线}
在空间,最原始的,也是最基本的概念就是“位置”。
通常,我们就用“点”来标记“位置”。例如在一张地图
上,我们就以小圆点来标记各地的位置(见图1.2).你
可能发现,在地图上北京用“$\bigstar$”,南京用“$\bigcirc$”印制
的,这只是为了把首都和地方城市区别开来。其实,北京、
南京的。“位置”与地图上印制的图形“$\bigstar$”或“$\bigcirc$”的形状
和大小是没有关系的。这样,仅仅考虑“位置”,的图形就是
点。在天象图上也是以小圆点来标记各星体的位置的(见图1.3)

在几何学的讨论中,我们用不同的大写字母$A,B,C,\ldots$
表示不同的点,如图1.4中的五个点,就在点旁分别标记
以$A$、$B$、$C$、$D$、$E$, 并分别读作点$A$、点$B$、点$C$、点
$D$、点$E$。



在日常生活中,我们经常需要从一个地方走到另一个地
方。例如,同学们早起上学,就得由自己的家所在的位置走
到学校所在的位置。因此,在空间第二个原始的基本概念就
要算是“通路”了。所谓“通路”,就是从一个位置移到
另一个位置的路线。通常在地图上,我们用线来标记各地之
间的种种通路,如铁路、公路等。在几何学的讨论中,“线”
就是表示通路的。它的直观含义就是:一个“动点”由一
个位置移动到另一个位置所走过的“路线”。如图1.5所
示,设$A$、$B$两点分别表示空间的两个位置,那么连结$A$、$B$
两点的可能通路是很多很多的。

在通常情况下,大家都希望所要走的通路愈短愈好,所
以很自然的问题就是:

“在所有连结$A$、$B$两点的各种通路中,哪一条通路最
短?”

光线的存在,直截了当地显示给我们下述空间的基本性
质:

“连结A、B两点的最短通路唯一存在,它就是连结$A$、
$B$两点的\textbf{直线段}”(在均匀介质中,光走直线\footnote{由光学实验,我们知道光线其实走着最省时间的通路,而并不
	是走着最短的通路,再者,光的速度是随着“介质”而定的,例如在
	真空中走的最快,在空气中速度则稍慢(愈稀薄则其速度愈近于真空
	者),在水中则速度更慢,因为通常我们总是在均匀介质中观察光
	线,所以光线的速度是个不变的常数。这样,最省时间的通路也就是
	最短的通路。这就是我们常见常用的事实:光线在均匀介质中走直
	线。})。

如图1.6所示,由$A$点射向$B$点的光线可以由$A$向$B$的方
向无限延伸;而由$B$点射向$A$点的光线也可以由$B$向$A$的方向
无限延伸,所以对于空间任意两点$A$、$B$, 不但存在着唯一
的最短通路“直线段$AB$”,而且也唯一地确定了一条把直线
段$AB$两端无限延长的直线,这条直线就叫做由$A$、$B$两点所
确定的\textbf{直线},通常称为“直线$AB$”,而直线段$AB$是直线$AB$
介于A、B两点之间的那一段。

归纳上面的讨论,我们可以作出如下的总结:








