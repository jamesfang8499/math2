\chapter{集合与简易逻辑}

 \section{集合}
\subsection{集合的概念}
我们在第一册中曾用到过集合的概念,如自然数所构成
的集合$\{1,2,3,\ldots\}$, 方程$x^2=4$的解所构成的集合$\{-2,2\}$
等等。那么,什么是集合呢?集合是数学中一个很根本的也
是很原始的概念,通常我们把一些确定的、彼此不同的“事
物”作为一个整体来考虑时,这个整体便说是一个\textbf{集合}。这
些事物叫做该集合的\textbf{元素}。例如:
某中学初二(一)班全体学生;小于100的全体质数;
一个生产队的全体社员;一个工厂的全部机器等等,都可分别构成一个集合。可见集合的概念是很简单
的。

对于集合这个概念,我们要注意以下几点:

第一,一个集合完全被它所含的元素所确定。至于集合
的元素之间是否具有某种相互关系,怎样排列,以及这些元
素所构成的集合是具有某种功效,单从集合的观点来看,
是一样的。例如,“一堆还没有组装的手表另件”和用“这
些另件组装好了的手表”是同一个集合。因为两者包含同样
的元素。因此,集合这个概念的要素是:\textbf{一个集合完全被它
所含的元素所确定}。

第二,集合是指构成集合的全体元素,而不是个别元
素。作为整体的集合和集合中的每个元素都是不同的。

例如,集合$A=\{a,b,c,d\}$, $A$代表的是字母$a$、$b$、
$c$、$d$的全体,而不是代表其中的个别字母,因此,作为字母
$a$、$b$、
$c$、$d$的整体的集合$A$与$A$中的个别元素如$a,b,c,d$不
能混为一谈。

第三,集合中所含的元素必须是“确定”的,是可以判
断的。例如,由“比较小的实数”的全体就不能构成一个集
合,因为到底什么叫做比较小的实数,没有判断的标准。但
是“比80小的实数”是完全可以确定的,这就有了检验一个
实数是否是这个集合的元素的标准。

任一几何图形,我们可以看作由点构成的,也就是可看
作点的集合。例如:

\textbf{圆}是同一平面上与一定点的距离等于定长的所有点的集
合(图2.1(1))。

\textbf{圆面}是同一平面上与一定点距离小于或等于定长的所有
点的集合。(图2.1(2))。

\begin{figure}[htp]
	\centering
	\begin{tikzpicture}[scale=.7]
		\begin{scope}
			\draw (0,0) circle (2);
\draw[ultra thick] (0,0)node[below]{$O$}--node[left]{$r$}(45:2);		
\node at (0,-3){(1)圆};	
		\end{scope}
		\begin{scope}[xshift=6cm]
			\draw[pattern=north west lines] (0,0) circle (2);
\draw[ultra thick] (0,0)node[below, fill=white]{$O$}--(45:2);	
			\node at (60:1.5) [left, rotate=45, fill=white]{$r$};
			
			\node at (0,-3){(2)圆面};
		\end{scope}
	\end{tikzpicture}	
	\caption{}
\end{figure}

我们通常用大写字母$A,B,C,\ldots$等表示某一个集合,
用小写字母$a,b,c,\ldots$表示集合的元素。如果$a$是集合$A$
的一个元素,我们就记为$a\in A$,
读作$a$属于$A$, 或说$a$是$A$中的一个元素。例如,$2\in\{2,3\}$,
表示$2$是集合$\{2,3\}$中的一个元素。

如果$a$不是集合$A$的元素,记作
$a\notin A$
读作$a$不属于$A$。

应该注意的是:几何图形中的元素“点”我们仍用大写
字母$A,B,C,\ldots$表示,这一点
请同学们务必注意,不要混
淆。如$X$点在直线$AB$上,也
可以说$X$点属于直线$AB$, 可
写成$X\in\text{直线}AB$. $Y$点不在直
线$AB$上,也可以说$Y$点不属于直线$AB$, 可写成$Y\notin\text{直线}
AB$(图2.2)。

\begin{figure}[htp]
	\centering
	\begin{tikzpicture}[scale=1]
\draw(0,0)--(6,0);
\draw (1,0)[fill=black]circle (1.5pt) node[below]{$A$};
\draw  (4.5,0)[fill=black]circle (1.5pt) node[below]{$B$};
\draw  (4,-.5)[fill=black]circle (1.5pt) node[below]{$Y$};
\draw  (3,0)[fill=black]circle (1.5pt) node[above]{$X$};
	\end{tikzpicture}	
	\caption{}
\end{figure}

\begin{ex}
\begin{enumerate}
\item 若$S$是
所有平方数的集合,试判定100至200之间哪些数
属于$S$.
\item 若$B$是所有英语元音字母所构成的集合,$A$是所有英语
辅音字母所构成的集合,试判定$a,b,c,d,e$这五个字
母分别属于哪一集合,又不属于哪一集合。
\end{enumerate}
\end{ex}

\subsection{集合的描述法}
决定一个集合的要素,就是它所含的元素,所以要描述
一个集合,也就是要描述它所含的是哪些元素。下面介绍两
种常用的集合描述法。

\subsubsection{列举法}
如果一个集合$A$只含有很少几个元素,那么可以直截了
当地把这个集合含有的所有元素逐一列举出来,并用大括号
$\{\quad \}$把它们括起来,这种描述法叫做\textbf{列举法}。

例如$\{0,1\}$是由0,1这两个元素所构成的集合;$\{+,-,\x,\div\}$表示由$+$、$-$、$\x$、$\div$四个运算符号所构成的
集合。用列举法描述集合时,描述方法与元素在括号内的排
列顺序无关,即$\{3,7,10\}$、$\{10,3,7\}$与$\{7,3,10\}$
都表示同一个集合。

\subsubsection{特征性质描述法}
当集合的元素稍多一些时,如小于100的质数所构成的
集合:$$\{2,3,5,7,11,13,17,19,23,29,31,37,
41,43,47,53,59,61,67,71,73,79,83,89,
97\}$$
逐一列举已是很麻烦的了,而对于含有无穷多个元素的
集合,例如全体整数所构成的集合,逐一列举它的元素更是
不可能的,这时我们可用某集合所含的元素的“特征性质”
去描述这个集合,这种方法叫做\textbf{特征性质描述法}。如:

\begin{enumerate}
\item 集合元素为
$\pm 2,\pm 4,\pm 6,\pm 8,\ldots,\pm 2n\ldots$的集
合,可描述为\{偶数}或\{能被2整除的数}。
\item 集合元素为$\pm 1,\pm 3,\pm 5,\pm 7,\ldots,\pm (2n+1)\ldots$的集合,可描述为\{奇数\}或\{被2除余1的数\}。
\item $\{-\sqrt{2},\sqrt{2}\}$,可描述为$\{\text{平方为2的数}\}$。
\item 圆面上不在圆上的点叫做圆内的点。在平面$P$上以$O$为
圆心,5厘米长为半径的圆内的点所成的集合,可描述为\{在
平面$P$上和点$O$的距离小于5厘米的点\}。	
\end{enumerate}

集合的特征性质描述法,常常采用下面更一般的形式:
\[A=\{x|\alpha\}\]
其中$x$表示集合$A$的任一元素,$x|\alpha$表示元素$x$具有特征性
质$\alpha$, 而$A=\{x|\alpha\}$则表示由所有具有性质$\alpha$的元素所构
成的集合$A$. 这样一来,上述各例又可表示如下:























